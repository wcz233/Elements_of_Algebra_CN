%\documentclass[10pt,oneside,a4paper]{book}
\documentclass{ctexbook} % 或 article, report 等
\usepackage{graphicx} % 如果需要插入图片
\usepackage{amsmath, amssymb} % 数学公式支持

%\usepackage{polynom}
%%%%%%%%%%%%%%%%% /longdiv %%%%%%%%%%%%%%%%%
\newcount\gpten % (global) power-of-ten -- tells which digit we are doing
\countdef\rtot2 % running total -- remainder so far
\countdef\LDscratch4 % scratch

\def\longdiv#1#2{%
 \vtop{\normalbaselines \offinterlineskip
   \setbox\strutbox\hbox{\vrule height 2.1ex depth .5ex width0ex}%
   \def\showdig{$\underline{\the\LDscratch\strut}$\cr\the\rtot\strut\cr
       \noalign{\kern-.2ex}}%
   \global\rtot=#1\relax
   \count0=\rtot\divide\count0by#2\edef\quotient{\the\count0}%\show\quotient
   % make list macro out of digits in quotient:
   \def\temp##1{\ifx##1\temp\else \noexpand\dodig ##1\expandafter\temp\fi}%
   \edef\routine{\expandafter\temp\quotient\temp}%
   % process list to give power-of-ten:
   \def\dodig##1{\global\multiply\gpten by10 }\global\gpten=1 \routine
   % to display effect of one digit in quotient (zero ignored):
   \def\dodig##1{\global\divide\gpten by10
      \LDscratch =\gpten
      \multiply\LDscratch  by##1%
      \multiply\LDscratch  by#2%
      \global\advance\rtot-\LDscratch \relax
      \ifnum\LDscratch>0 \showdig \fi % must hide \cr in a macro to skip it
   }%
   \tabskip=0pt
   \halign{\hfil##\cr % \halign for entire division problem
     $\quotient$\strut\cr
     #2$\,\overline{\vphantom{\big)}%
     \hbox{\smash{\raise3.5\fontdimen8\textfont3\hbox{$\big)$}}}%
     \mkern2mu \the\rtot}$\cr\noalign{\kern-.2ex}
     \routine \cr % do each digit in quotient
}}}
%%%%%%%%%%%%%%%%% end /longdiv %%%%%%%%%%%%%%%%%

%导言区
%宏包
\usepackage{tikz}
\usepackage{tabularray}
\usepackage[a4paper, margin=1in]{geometry} % 设置 1 英寸边距
\usepackage{float} % 引入 float 宏包,用于定位表格位置[H]

% 定义无编号的章节标题
\CTEXsetup[name={第,章}]{section}
\CTEXsetup[name={第,节}]{subsection}
\CTEXsetup[number={\chinese{section}}]{section}
\CTEXsetup[number={\chinese{subsection}}]{subsection}

\begin{document}
%正文内容

\title{Elements of Algebra}
\author{Leonhard Euler}
\date{\today}
\maketitle

\tableofcontents
\part{对确定量的分析}
\section{对简单量的不同计算方法}
\subsection{关于数学的一般概述}


1. 凡是能够增加或减少的东西, 称为量或数量。
因此, 金钱是一种数量, 因为我们可以增加或减少它。
重量以及其他类似性质的事物也是如此。


2. 根据这一定义, 很明显, 不同种类的量必然非常多, 
以至于难以一一列举。这就是数学各个分支的起源, 每个分支都专注于某种特定类型的量。
数学总体上是研究数量的科学, 或者说, 是研究如何测量数量的科学。


3. 现在, 我们无法测量或确定任何数量, 除非先将同类中的某个已知数量作为参照, 并指出它们之间的相互关系。
例如, 如果要确定一笔钱的数量, 我们需要选择某种已知的货币单位, 
如路易金币、克朗、杜卡特或其他硬币, 然后计算这笔钱中包含多少个这样的单位。
同样, 如果要确定重量的数量, 我们需要选择某个已知的重量单位, 
如磅或盎司, 然后计算这种单位在我们试图确定的重量中包含多少次。
如果我们希望测量某种长度或距离, 则可以使用已知的长度单位, 比如一英尺。


4. 因此, 各种类型的量的确定或测量可以归结为以下方法:任意选择与待测量同类的一已知量, 
将其作为标准或单位;然后, 确定待测量与该已知标准之间的比例。
这个比例总是用数字表示, 因此, 数字只是一个量与另一个任意设定为单位的量之间的比例。


5. 由此可见, 所有的量都可以用数字表示, 
并且所有数学科学的基础必须建立在对数字科学的完整研究和对各种可能计算方法的精确考察之上。
这一数学的基础部分被称为 \textbf{\textit{分析学(Analysis)}} 或 \textbf{\textit{代数学(Algebra)}} 。


\subsection{对加号“ $+$ ”和减号“ $-$ ”的解释}


8. 当我们需要将一个给定的数字加到另一个数字上时, 
	用符号“ $+$ ”表示, 这个符号放在第二个数字前面, 读作“加”。
	例如, $5 + 3$ 表示要将 $3$ 加到 $5$ 上, 在这种情况下, 众所周知结果是 $8$。
	同样, $12 + 7$ 等于 $19$; $25 + 16$ 等于 $41$; $25 + 41$ 的和是 $66$, 等等。
\par
9. 我们也使用相同的加号“$+$”来连接几个数字。
	例如, $7 + 5 + 9 + 9$ 表示将 $5$ 和 $9$ 分别加到 $7$ 上, 
	结果是 $21$。因此, 读者应该能够理解以下表达式的含义: 
\[
		8 + 5 + 13 + 11 + 1 + 3 + 10
\]\par
	即这些数字的总和是 $51$。
\par
10. 这一切都显而易见; 我们只需提到, 在代数学中, 为了对数字进行泛化, 	
	我们用字母来表示它们, 如 $a$、$b$、$c$、$d$ 等。
	因此, 表达式 $a + b$ 表示由 a 和 b 代表的两个数字的总和, 
	而这些数字可以非常大, 也可以非常小。同样, 
$f + m + b+ x$
表示由这些四个字母代表的数字的总和。
\par
	因此, 如果我们知道这些字母所代表的数字, 就可以随时通过算术来找到这些表达式的总和或值。
\par
11. 相反, 当需要从一个给定数字中减去另一个数字时, 这个操作用符号“$-$”表示, 读作“减”,
 并放在要被减去的数字前。例如, $8 - 5$ 表示从 $8$ 中减去 $5$, 结果是 $3$。同样, $12 - 7$ 等于 $5$; $20 - 14$ 等于 $6$, 等等。
\par
12. 有时, 我们可能需要从一个数字中减去多个数字。例如: 
\[
	50 - 1 - 3 - 5 - 7 - 9
\]
这表示首先从 $50$ 中减去 $1$, 剩下 $49$; 然后从这个余数中减去 $3$, 剩下 $46$; 
再减去 $5$, 剩下 $41$; 减去 $7$, 剩下 $34$; 最后, 从中减去 $9$, 剩下 $25$: 这个最终的余数就是该表达式的值。
但由于 $1$、$3$、$5$、$7$、$9$ 这几个数字都要被减去, 因此可以直接将它们的总和 $25$ 一次性从 $50$ 中减去, 结果仍然是 $25$。
\par
13. 对于包含加号“$+$”和减号“$-$”的类似表达式, 其值也容易确定。例如: 
\[
	12 - 3 - 5 + 2 - 1 = 5
\]
	我们只需分别计算带有加号“$+$”的数字的总和, 
	并从中减去带有减号“$-$”的数字的总和。
	因此, $12$ 和 $2$ 的总和是 $14$; 
	而 $3$、$5$ 和 $1$ 的总和是 $9$; 
	于是, 从 $14$ 中减去 $9$, 剩下 $5$。
\par
14. 从这些例子可以看出, 
	只要保留每个数字前面的正确符号, 
	数字的书写顺序完全无关紧要且是任意的。
	我们可以同样正确地将前一条中的表达式排列为: \par
	   $12 + 2 - 5 - 3 - 1$, \par
	或 $2 - 1 - 3 - 5 + 12$, \par
	或 $2 + 12 - 3 - 1 - 5$, \par
	或者其他不同的顺序; 在最初的排列中, 需要注意, 假定数字 $12$ 前面有一个加号“$+$”。
\par
15. 如果为了将这些运算一般化, 我们使用字母代替实际数字, 这并不会带来更多困难。例如, 很明显, 表达式
\[
	a - b - c + d - e
\]
	表示我们有由 $a$ 和 $d$ 表示的数字, 
	而需要从这些数字或它们的和中减去由字母 $b$、$c$ 和 $e$ 表示的数字, 
	它们前面带有符号“$-$”。
\par
16. 因此, 必须严格注意每个数字前所带的符号。
	在代数学中, 简单数量是考虑其前面所带符号的数字。
	进一步说, 我们称那些前面带有符号“$+$”的量为正量, 
	而那些受符号“$-$”影响的量为负量。
\par
17. 我们通常计算一个人财产的方法恰好说明了上述情况。
	我们用正数来表示一个人实际拥有的财产, 
	并使用或默认加号“$+$”; 
	而他的债务则用负数表示, 
	或者使用符号“$-$”。
	因此, 当说某人拥有 $100$ 克朗, 
	但欠债 $50$ 克朗时, 
	这意味着他的实际财产是 $100 - 50$; 或者换个形式, $+100 - 50$, 即 $50$。
\par
18. 由于负数可以被视为债务, 而正数代表实际拥有的财产, 我们可以说负数小于零。
因此, 当一个人一无所有却欠 $50$ 克朗时, 显然他比一无所有还要少 $50$ 克朗。
因为即使有人赠送他 $50$ 克朗来偿还债务, 他也只是达到零的水平, 尽管实际上比以前富有。
\par
19. 因此, 正数无可争议地大于零, 而负数则小于零。我们通过在 $0$ 上加 $1$ 来得到正数, 即在零的基础上加 $1$; 
并不断从单位数开始增加。这就是所谓自然数序列的起源, 其主要项如下: 
\[
0, +1, +2, +3, +4, +5, +6, +7, +8, +9, +10, \ \text{依此类推至无穷。}
\] 
但如果我们不是通过连续加 $1$ 来延续这一序列, 而是反方向进行, 即不断减去 $1$, 就会得到以下负数序列: 
\[
0, -1, -2, -3, -4, -5, -6, -7, -8, -9, -10, \ \text{依此类推至无穷。}
\] 
20. 这些数字, 无论是正数还是负数, 都被统称为整数, 它们因此要么大于零, 要么小于零。
我们称它们为整数, 是为了将其与分数以及其他一些种类的数字区分开来, 这些将在后面讨论。
例如, $50$ 比 $49$ 多 $1$ 个完整的单位, 很容易理解, 在 $49$ 和 $50$ 之间可能有无数个中间数, 
这些数都大于 $49$, 但又都小于 $50$。我们只需想象两条线, 
一条长 $50$ 英尺, 另一条长 $49$ 英尺, 很明显, 可以画出无数条线, 它们都比 $49$ 英尺长, 但又比 $50$ 英尺短。
\par
21. 在整个代数学中, 对负量的准确理解极为重要。然而, 我在这里仅指出, 所有如下形式的表达式: \par
	$+1 - 1, +2 - 2, +3 - 3, +4 - 4$, 等等\par
	都等于 $0$, 或零。而表达式
	$+2 - 5$ 等于 -3, 
	因为如果一个人有 $2$ 克朗, 
	但欠 $5$ 克朗, 他不仅一无所有, 还欠 $3$ 克朗。\par
	同样, $7 - 12$ 等于 $-5$, 
	而 $25 - 40$ 等于 $-15$。  
\par
22. 同样的观察结论也适用于使用字母代替数字以使表达式更加一般化的情况。
	例如, $+a - a$ 的值始终为 $0$ 或零。但如果我们想知道 $+a - b$ 的值, 需要考虑两种情况。\par
	第一种情况是 $a$ 大于 $b$, 此时应从 $a$ 中减去 $b$, 
	其余数(其前面带有符号“$+$”或默认带有符号“$+$”)即为所求的值。\par
	第二种情况是 $a$ 小于 $b$, 此时应从 $b$ 中减去 $a$, 
	将余数变为负数, 在其前面加上符号“$-$”, 即为所求的值。  
\par

\subsection{关于简单量的乘法}
\par
23. 当需要将两个或多个相等的数相加时, 可以简化其和的表达式。例如: 
\begin{gather*}
	a + a \ \text{等于}\  2 \times a, \\
	a + a + a \ \text{等于}\  3 \times a, \\
	a + a + a + a \ \text{等于}\  4 \times a, \text{依此类推}; 
\end{gather*}
其中 $\times$ 是乘法的符号。通过这种方式, 我们可以形成对乘法的概念, 并且需要注意: \par
\begin{gather*}
\begin{aligned}
&1 \times a \ \text{表示}\  2 \ \text{次}\  a, \ \text{或}\  a \ \text{的两倍}\ ;\\
&2 \times a \ \text{表示}\  3 \ \text{次}\  a, \ \text{或}\  a \ \text{的三倍}\ ;\\
&3 \times a \ \text{表示}\  4 \ \text{次}\  a, \ \text{或}\  a \ \text{的四倍}\ , \ \text{依此类推。}
\end{aligned}
\end{gather*}
\par
24. 因此, 如果用字母表示的一个数需要乘以另一个数, 只需将该数写在字母前。例如: \par
\begin{gather*}
\begin{aligned}
&a \ \text{乘以}\  20 \ \text{表示为}\  20a,\\
&b \ \text{乘以}\  30 \ \text{表示为}\  30b, \ \text{依此类推。}
\end{aligned}
\end{gather*}
显然, $1c$, 或 c 只取一次, 与 c 相同。
\par
25. 此外, 再将这些积乘以其他数也非常容易。例如: \par
\begin{gather*}
\begin{aligned}
&2 \ \text{次}\  3a, \ \text{或}\  3a \ \text{的两倍}\ , \ \text{是}\  6a;\\
&3 \ \text{次}\  4b, \ \text{或}\  4b \ \text{的三倍}\ , \ \text{是}\  12b;\\
&5 \ \text{次}\  7x \ \text{是}\  35x\ \text{。}
\end{aligned}
\end{gather*}
这些积还可以随意继续乘以其他数。
\par
26. 当要乘的数也用字母表示时, 将其直接写在另一个字母前。
例如, 将 $b$ 乘以 $a$, 积写作 $ab$; 
而 $pq$ 则是将 $q$ 乘以 $p$ 的积。
同样, 如果再将 $pq$ 乘以 $a$, 结果是 $apq$。
\par
27. 此处还可以进一步指出, 
字母相乘的顺序无关紧要。例如, $ab$ 与 $ba$ 相同, 
因为 $b$ 乘以 $a$ 与 $a$ 乘以 $b$ 是相同的。要理解这一点, 
只需将 $a$ 和 $b$ 替换为已知数字, 例如 $3$ 和 $4$; 
显而易见, $3$ 乘以 $4$ 与 $4$ 乘以 $3$ 是相同的。
\par
28. 不难看出, 当我们将数字代入这些连乘的字母表达式时, 
不能简单地将它们写在一起。例如, 如果将 $3$ 和 $4$ 相乘后写作 $34$, 
就得到了 $34$, 而不是 $12$。\par
因此, 当需要将普通数字相乘时, 
必须用乘号“ $\times$ ”或一个点“ $\cdot$ ”来分隔它们。\par
故,$3 \times 4,$ 或 $3 \cdot 4$ 意为 $3$ 乘 $4$, 也就是 $12$ ; 
同样, $1 \times 2$ 等于 $2$; $1 \times 2 \times 3$ 等于 $6$。\par
类似地 $1 \times 2 \times 3 \times 4 \times 56$ 等于 $1344$ , $1 \times 2 \times 3 \times 4 \times 5 \times 6 \times 7 \times 8 \times 9 \times 10$ 等于 $3628800$, 依此类推。
\par
29. 同样, 我们可以求出形如 $5 \cdot 7 \cdot 8 \cdot abcd$ 的表达式的值。
这表示 $5$ 必须乘 $7$, 然后将得到的积再乘 $8$; 
接下来, 将这三个数的积依次乘 $a$、$b$、$c$ 和 $d$。还可以注意到, 我们可以将 $5 \cdot 7 \cdot 8$ 替换为它们的积 $280$, 因为将 $5$ 乘 $7$ 的积 $35$ 再乘 $8$, 即得 $280$。
\par
30. 由两个或多个数的乘法所产生的结果称为\textbf{\textit{积(products)}}, 而这些数或单独的字母称为\textbf{\textit{因子(factors)}}。
\par
31. 迄今为止, 我们只考虑了正数的情况。毫无疑问, 这些正数所产生的积也是正的: $+b$ 乘以 $+a$ 必然得到 $+ab$。
但我们还需要分别研究 $+a$ 乘以 $-b$, 以及 $-a$ 乘以 $-b$ 时的结果。
\par
32. 让我们从 $-a$ 乘以 3 或 +3 开始。由于 $-a$ 可以看作是债务, 很明显, 如果将这笔债务算作三次, 
那么它将变为原来的三倍, 因此所得的积为 $-3a$。
同样, 如果 $-a$ 乘以 $+b$, 我们将得到 $-ba$, 或者换个形式, $-ab$。
因此, 我们可以得出结论: 如果一个正的数量与一个负的数量相乘, 所得的积是负的; 并且可以归纳出以下规则: \par
$+$ 乘以 $+$ 得 $+$ , \par
相反 $+$ 乘以 $-$ , \par
或 $\ \ \ -$ 乘以 $+$ 得 $-$。
\par
33. 接下来, 我们解决 $-$ 乘以 $-$ 的情况, 例如 $-a$ 乘以 $-b$。
首先从字母上看, 积显然是 $ab$, 但其前面的符号是 $+$ 还是 $-$ 尚不确定, 
我们只知道它必须是其中之一。现在, 我说它不可能是符号 $-$: 因为 $-a$ 乘以 $+b$ 得 $-ab$, 
而 $-a$ 乘以 $-b$ 不可能产生与 $-a$ 乘以 $+b$ 相同的结果, 而必须产生相反的结果, 即 $+ab$。
因此, 我们得出以下规则:\par
 $-$ 乘以 $-$ 得 $+$ , \par
 这与 $+$ 乘以 $+$ 的结果相同。
\par
34. 我们可以用以下简洁的方式表达这些规则: \par
\textbf{\textit{相同符号相乘, 得 $+$; 相异符号相乘, 得 $-$。}} \par
例如, 当需要将以下数相乘时: $+a$、$-b$、$-c$、$+d$, 我们首先计算 $+a$ 乘以 $-b$, 
得到 $-ab$; 接着, 用 $-c$ 乘以该积, 得到 $+abc$; 最后, 将 $+d$ 乘以该积, 得到 $+abcd$。
\par
35. 在解决符号问题后, 我们接下来要说明如何相乘本身是积的数。
例如, 如果将数 ab 乘以数 cd, 所得积为 abcd。这可以通过先将 ab 乘以 c, 然后再将所得结果乘以 d 来实现。
或者, 如果我们需要将 36 乘以 12, 由于 12 等于 3 乘以 4, 我们只需先将 36 乘以 3, 得 108, 
再将 108 乘以 4, 得出 36 乘以 12 的结果, 即 432。
\par
36. 如果我们想将 $5ab$ 乘以 $3cd$, 可以将其写作 $3cd \times 5ab$。
然而, 由于在此情况下乘数的顺序无关紧要, 按照惯例, 更好的做法是将普通数字写在字母前, 
并将积表示为 $5 \times 3abcd$, 即 $15abcd$, 因为 5 乘以 3 得 15。同样, 
如果我们需要将 $12pqr$ 乘以 $7xy$ , 我们会得到 $12 \times 7pqrxy$ , 即 $84pqrxy$ 。
\subsection{关于整数的性质及其因子}
\par
37. 我们已经指出, 积是由两个或多个乘数相乘得到的, 
这些乘数被称为因子。因此, 数字 $a$、$b$、$c$、$d$ 是积 $a$bc$d$ 的因子。
\par
38. 如果我们将所有整数视为由两个或多个数相乘而成的积, 那么很快就会发现, 
有些数不能由这样的乘法得出, 因此没有任何因子; 而另一些数可以由两个或多个数相乘得出, 
因此有两个或多个因子。例如, $4$ 是由 $2 \times 2$ 生成的; $6$ 是由 $2 \times 3$ 生成的; 
$8$ 是由 $2 \times 2 \times 2$ 生成的; $27$ 是由 $3 \times 3 \times 3$ 生成的; 
$10$ 是由 $2 \times 5$ 生成的, 等等。
\par
39. 另一方面, 数字 $2$、$3$、$5$、$7$、$11$、$13$、$17$ 等不能以相同的方式用因子表示, 
除非我们使用 $1$, 例如将 $2$ 表示为 $1 \times 2$。但由于任何数与$1$相乘仍为其自身, 因此不应将$1$视为因子。  
因此, 所有不能用因子表示的数字, 
例如 $2$、$3$、$5$、$7$、$11$、$13$、$17$ 等, 
被称为\textbf{\textit{素数(simple)}} 或 \textbf{\textit{质数(prime numbers)}}; 
而其他可以用因子表示的数字, 例如 $4$、$6$、$8$、$9$、$10$、$12$、$14$、$15$、$16$、$18$ 等, 
被称为\textbf{\textit{合数(composite)}}。
\par
40. 质数或素数由于不是由两个或多个数相乘得到的, 因此值得特别注意。
还有一点特别值得关注的是, 如果按顺序写出这些质数, 如下: 
\[
2, 3, 5, 7, 11, 13, 17, 19, 23, 29, 31, 37, 41, 43, 47, \ \text{等等, }
\]
我们无法发现任何规律; 它们之间的差距有时较大, 有时较小。
至今尚无人能够确定它们是否遵循某种特定的规律。
\par
41. 所有可以用因子表示的合数, 都是由上述质数生成的。也就是说, 
它们的所有因子都是质数。例如, 如果我们找到一个不是质数的因子, 
它总是可以被分解为两个或多个质数。例如, 将数字 $30$ 表示为 $5 \times 6$, 
但由于 $6$ 不是质数, 而是由 $2 \times 3$ 生成的, 我们可以将 $30$ 表示为 $5 \times 2 \times 3$, 
或 $2 \times 3 \times 5$, 即完全由质数组成的因子。
\par
42. 如果我们现在考虑那些可以分解为质因子的合数, 会发现它们之间存在很大的差异。
例如, 我们会发现有些数只有两个因子, 有些数有三个因子, 还有更多的有更多因子。例如:   
\begin{gather*}
	\begin{aligned}
		4 &= 2 \times 2, \\
		6 &= 2 \times 3, \\
		8 &= 2 \times 2 \times 2, \\
		9 &= 3 \times 3, \\
		10 &= 2 \times 5, \\
		12 &= 2 \times 3 \times 2, \\
		14 &= 2 \times 7, \\
		15 &= 3 \times 5, \\
		16 &= 2 \times 2 \times 2 \times 2, \\
	\end{aligned}
\end{gather*}
等等。
\par
43. 因此, 可以很容易地找到一种方法来分析任意数字或将其分解为其质因子。
例如, 给定数字 $360$, 我们首先将其表示为 $2 \times 180$。\par
现在, $180$ 等于 $2 \times 90$, 而  
\begin{gather*}
	\begin{aligned}
		90 &= 2 \times 45 \\
		45 &= 3 \times 15 \\
		15 &= 3 \times 5 \\
	\end{aligned}
\end{gather*}
所以, 数字 $360$ 可以表示为这些质因子的乘积: $2 \times 2 \times 2 \times 3 \times 3 \times 5$, 
因为这些数相乘的结果是 $360$。

44. 这表明质数不能被其他数字整除; 
另一方面, 复合数的质因子可以通过寻找能将这些复合数整除的质数来最方便、最可靠地找到。
而要做到这一点, 就需要用到\textbf{\textit{除法(Division)}}。因此, 我们将在下一章中解释这一操作的规则。


\subsection{关于简单量的除法}
\par
45. 当将一个数分成两份、三份或更多相等部分时, 
可以通过\textbf{\textit{除法(div-~ision)}}来完成, 
这使我们能够确定其中一个部分的大小。
例如, 当我们希望将数字 $12$ 分成三份相等的部分时, 通过除法我们得知每部分的大小为 $4$。
在这种运算中使用以下术语: 要被分解或除去的数称为\textbf{\textit{被除数(dividend)}}; 
要找的那个大小的数量称为\textbf{\textit{除数(divisor)}}; 
通过除法确定的其中一份的大小称为\textbf{\textit{商(quotient)}}。因此, 在上述例子中: 
\begin{gather*}
	\begin{aligned}
12 \ \text{是被除数, }  
3 \ \text{是除数, } 
4 \ \text{是商。}
	\end{aligned}
\end{gather*}

46. 由此可见, 如果将一个数除以 $2$, 
或者将其分成两等份, 则其中一份(即商)取两次就正好等于原来的数。
同样地, 如果将一个数除以 $3$, 那么商取三次也会得到原来的数。
一般来说, 将商乘以除数, 必然能够重新得到被除数。

47. 正因如此, \textbf{\textit{除法(division)}}被称为一种能够帮助我们找到一个数量(即商)的运算规则, 
这个数乘以除数可以准确地得到被除数。
例如, 如果将 $35$ 除以 $5$, 就是说我们需要找到一个数, 这个数乘以 $5$ 会得出 $35$。
这个数是 $7$, 因为 $5$ 乘 $7$ 等于 $35$。这种推理的表达方式是: “$7$ 在 $35$ 中能包含 $5$ 次”, 并且“$5$ 乘 $7$ 等于 $35$”。

48. 因此, 被除数可以看作是一个乘积, 其中一个因子是除数, 
另一个因子是商。例如, 如果我们要将 $63$ 除以 $7$, 我们试图找到一个乘积, 
使得 $7$ 作为其中一个因子, 且另一个因子乘以 $7$ 时恰好等于 $63$ 。此时 $7\times9$ 恰好是这样的乘积, 
而 $9$ 也正是 $63$ 除以 $7$ 时得到的商。

49. 一般而言, 如果我们将数字 $ab$ 除以 $a$, 很明显商是 $b$, 因为 $a$ 乘 $b$ 得到被除数 $ab$。
同样地, 如果我们将 $ab$ 除以 $b$, 商就是 $a$。在所有可以提出的除法例子中, 如果用被除数除以商, 
我们将再次得到除数。例如, $24$ 除以 $4$ 得 $6$, 同样 $24$ 除以 $6$ 又得 $4$。

50. 由于整个运算过程实质上是将被除数表示为两个因子的乘积, 其中一个因子等于除数, 
另一个因子等于商, 以下例子将更容易理解。我说, 首先, 
被除数 $abc$ 除以 $a$, 得到的商是 $bc$, 因为 $a$ 乘 $bc$ 得到 $abc$。同样地, $abc$ 除以 $b$, 
商是 $ac$, 并且 $abc$ 除以 $ac$ 将得到 $b$。
很明显, $12mn$ 除以 $3m$ 得到 $4n$; $3m$ 乘以 $4n$ 得到 $12mn$。
而若同样这个数 $12mn$ 除以 $12$ , 我们也将得到商 $mn$。

51. 由于每个数 $a$ 都可以表示为 $1a$ 或 $a$, 很明显, 如果我们将 $a$ 或 $1a$ 除以 $1$, 
商将是相同的数 $a$。相反地, 如果将相同的数 $a$ 或 $1a$ 除以 $a$, 商将是 $1$。

52. 然而, 有时我们无法将被除数表示为两个因子的乘积, 其中一个因子等于除数。在这种情况下, 无法按照我们描述的方法完成除法运算。
例如, 将 $24$ 除以 $7$, 很明显 $7$ 不是 $24$ 的因子, 因为 $7\times3$ 仅等于 $21$, 
小于 $24$, 而 $7\times4$ 等于 $28$, 大于 $24$。
因此, 可以得出结论, 那个商必须大于 $3$ 且小于 $4$。为了更准确地确定商, 
我们需要使用另一种数——\textbf{\textit{分数(fraction)}}, 我们将在后续章节中讨论。

53. 在使用分数之前, 通常情况下, 人们会取最接近真实商的整数, 同时也关注留下的余差部分。
因此, 我们说, $7$ 在 $24$ 中能包含 $3$ 次, 残留部分是 $3$, 因为 $7$ 乘以 $3$ 仅得 $21$, 比 $24$ 小 $3$。  
类似地, 可以用同样的方式考虑以下例子:   
\[
\longdiv{34}{6}
\]
也就是说, 除数是$6$, 被除数是$34$, 被除数是$5$, 余数是$4$。
\[
\longdiv{41}{9}
\]
这里除数是$9$, 被除数是$41$, 商是$4$, 余数是$5$。
\par
在有余数的例子中,应遵守以下规则。

54. 用商乘以除数, 然后将余数加到所得的积上, 结果应等于被除数。
这种方法可以用来验证除法计算是否正确。例如, 在上述第一个例子中, 
将 $6$ 乘以 $5$ 得 $30$, 再加上余数 $4$, 结果是 $34$, 即被除数。在第二个例子中, 
将除数 $9$ 乘以商 $4$ 得 $36$, 再加上余数 $5$, 结果是 $41$, 即被除数。

55. 最后, 需要注意关于符号 \textbf{\textit{“$+$”正号(plus)}} 和 \textbf{\textit{“$-$”负号(minus)}}的问题。
如果将 $+ab$ 除以 $+a$, 商是 $+b$, 这是显然的。如果将 $+ab$ 除以 $-a$, 商是 $-b$, 因为 $-a$ 乘以 $-b$ 得到 $+ab$。同样地, 
如果将 $-ab$ 除以 $+a$, 商是 $-b$, 因为 $+a$ 乘以 $-b$ 得到 $-ab$。最后, 如果将 $-ab$ 除以 $-a$, 商是 $+b$, 因为 $-a$ 乘以 $+b$ 得到 $-ab$。

56. 因此, 对于符号 “$+$” 和 “$-$”, 除法遵循与乘法相同的规则, 即: 
\begin{gather*}
\begin{aligned}
&+ \ \text{除以}\  + \ \text{得}\  +; \\  
&+ \ \text{除以}\  - \ \text{得}\  -; \\
&- \ \text{除以}\  + \ \text{得}\  -; \\
&- \ \text{除以}\  - \ \text{得}\  +.
\end{aligned}
\end{gather*}
简而言之, 相同符号相除得正, 不同符号相除得负。

57. 例如, 将 $18pq$ 除以 $-3p$, 商是 $-6q$。此外,   
\begin{gather*}
\begin{aligned}
 -30xy \ \text{除以} +6y, \ \text{得} -5x;   
 -54abc \ \text{除以} -9b, \ \text{得} +6ac;   
\end{aligned}
\end{gather*}
在最后一个例子中, $-9b$ 乘以 $+6ac$ 等于 $-54abc$。  
关于简单量的除法已讨论足够多, 接下来我们将在补充一些关于数的性质及其因子的说明后, 转而解释\textbf{\textit{分数(fractions)}}。

\subsection{整数在其因数方面的性质}
\par
58. 如前所述, 有些数可以被某些因数整除, 而另一些则不能。
为了更深入地了解整数之间的区别, 我们需要仔细观察这种差异。
这不仅包括区分能被因数整除的数与不能被整除的数, 还包括研究后者在除法中所余的余数。为此, 我们将依次考察以下因数: 
\[
	2, 3, 4, 5, 6, 7, 8, 9, 10, \dots
\]
59. 首先, 以$2$为因数。能被$2$整除的数有: 
\[
	2, 4, 6, 8, 10, 12, 14, 16, 18, 20, \ \text{等等。}\  
\]
这些数显然总是以$2$为增量递增。这些数被称为\textbf{\textit{偶数(even numbers)}}。  
然而, 还有一些数, 如: 
\[
1, 3, 5, 7, 9, 11, 13, 15, 17, 19, \ \text{等等}\ , 
\]
它们与上述偶数始终相差$1$, 且无法被$2$整除, 而在除以$2$时总余$1$, 这些数被称为\textbf{\textit{奇数(odd numbers)}}。
\par
偶数可以用通用表达式 $2a$ 表示, 因为将整数 $1, 2, 3, 4, 5, 6, 7, $等依次代入 $a$, 即可生成所有偶数。
因此, 奇数可以用通式 $2a + 1$ 表示, 因为 $2a + 1$ 比偶数 $2a$ 大1。
\par
60. 其次, 设因数为3。能被3整除的数有: 
\[
	3, 6, 9, 12, 15, 18, 21, 24, 27, 30, \ \text{等等。}\ 
\]
这些数可以用表达式 $3a$ 表示, 因为 $3a$ 除以$3$时, 商为 $a$, 且没有余数。  
其他不能被$3$整除的数在除以$3$时会余$1$或$2$, 因此分为两类:   
第一类, 除以$3$余$1$的数有: 
\[
	1, 4, 7, 10, 13, 16, 19, \ \text{等等}\ , 
\]
可以用 $3a + 1$ 表示; 
第二类, 除以$3$余$2$的数有: 
\[
	2, 5, 8, 11, 14, 17, 20, \ \text{等等}\ , 
\]
可以用 $3a + 2$ 表示。  
因此, 所有整数都可以用以下三种形式之一表示: 
$3a$、$3a + 1$或$3a + 2$。

61. 再次, 设因数为$4$。能被$4$整除的数有: 
\[
	4, 8, 12, 16, 20, 24, \ \text{等等}\ , 
\]
这些数以$4$为增量递增, 可以用表达式 $4a$ 表示。  
而其他不能被$4$整除的数在除以$4$时, 可以有以下三种余数:   
第一类, 余数为$1$的数有: 
\[
	1, 5, 9, 13, 17, 21, 25, \ \text{等等}\ , 
\]
可以用 $4a + 1$ 表示; 
第二类, 余数为$2$的数有: 
\[
	2, 6, 10, 14, 18, 22, 26, \ \text{等等}\ , 
\]
可以用 $4a + 2$ 表示;   
第三类, 余数为$3$的数有: 
\[
	3, 7, 11, 15, 19, 23, 27, \ \text{等等}\ , 
\]
可以用 $4a + 3$ 表示。  

因此, 所有整数都可以用以下四种形式之一表示: 
\[
	4a, 4a + 1, 4a + 2, 4a + 3.
\]
62. 当以$5$为因数时, 情况与之前的讨论大致相同。所有能被$5$整除的数都可以用表达式 $5a$ 表示, 而不能被$5$整除的数则可以归为以下四种形式之一:   
\[
	5a + 1, 5a + 2, 5a + 3, 5a + 4.   
\]
同样的方法可以应用于更大的因数。

63. 这里有必要回顾一下关于将数分解为其简单因数的内容。  
任何包含因数$2$、$3$、$4$、$5$、$7$或其他数的整数, 必然能被这些数整除。例如, $60$可以表示为 $2 \times 2 \times 3 \times 5$, 因此显然$60$可以被$2$、$3$和$5$整除。

64. 此外, 通式 $abcd$ 不仅可以被 $a$、$b$、$c$ 和 $d$ 整除, 还可以被以下任意因子组合整除: \par
两个因子的组合: $ab$、$ac$、$ad$、$bc$、$bd$、$cd$; \par
三个因子的组合: $abc$、$abd$、$acd$、$bcd$; \par
以及所有因子的组合 $abcd$, 也就是其自身。\par
由此可见, $60$(即 $2 \times 2 \times 3 \times 5$)不仅可以被这些简单因数整除, 
还可以被任意两个简单因数的乘积整除, 如$4$、$6$、$10$、$15$; 也可以被任意三个因数的乘积整除, 如$12$、$20$、$30$; 最后, 还可以被$60$自身整除。

65. 因此, 当我们将一个任意选定的数分解为其简单因数后, 便可以轻松列出所有能整除它的数。  
具体方法是: 
首先将这些简单因数逐一列出; 然后将它们两两相乘, 接着三三相乘, 四四相乘, 依此类推, 直到得到该数本身。

66. 此处需要特别注意以下两点:
\begin{gather*}
\begin{aligned}
	&\ \text{每个数都可以被}\ 1\ \text{整除}\ ; \\
	&\ \text{每个数都可以被它本身整除。}\ 
\end{aligned}
\end{gather*}
因此, 每个数至少有两个因数(factors)或约数(divisors), 即该数本身和\textbf{\textit{1(unity)}}。
然而, 对于那些除了$1$和它本身之外没有其他因数的数, \par
我们将其归为\textbf{\textit{简单数(simple)}}或\textbf{\textit{素数(prime numbers)}}。  

除这些素数外, 其他所有数除了$1$和自身之外, 还具有其他因数。这可以从以下表格中看出, 在每个数下面列出了它的所有因数: 
\begin{table}[H]
	\centering
	\begin{tblr}{
	  row{odd} = {c},
	  row{2} = {c,t},
    vlines = {},
	  hlines = {}
	}
	1  & 2      & 3      & 4         & 5      & 6            & 7      & 8            & 9         & 10            & 11      & 12                  & 13      & 14            & 15            & 16               & 17      & 18                  & 19      & 20                   \\
	1  & {1\\2} & {1\\3} & {1\\2\\4} & {1\\5} & {1\\2\\3\\6} & {1\\7} & {1\\2\\4\\8} & {1\\3\\9} & {1\\2\\5\\10} & {1\\11} & {1\\2\\3\\4\\6\\12} & {1\\13} & {1\\2\\7\\14} & {1\\3\\5\\15} & {1\\2\\4\\8\\16} & {1\\17} & {1\\2\\3\\6\\9\\18} & {1\\19} & {1\\2\\4\\5\\10\\20} \\
	1  & 2      & 2      & 3         & 2      & 4            & 2      & 4            & 3         & 4             & 2       & 6                   & 2       & 4             & 4             & 5                & 2       & 6                   & 2       & 6                    \\
	P. & P.     & P.     &           & P.     &              & P.     &              &           &               & P.      &                     & P.      &               &               &                  & P.      &                     & P.      &                      
	\end{tblr}
\end{table}
表中以 “P.” 标注的为素数。

67. 最后, 需要注意的是, $0$, 或\textbf{\textit{零(nothing)}}, 可以被视为一个具有特殊性质的数: 它能被所有可能的数整除。  
无论用何数 $a$ 去除$0$, 其商始终为$0$。因为任何数乘以\textbf{\textit{零(nothing)}}都会得到零, 所以 $0 \times a$ 或 $0a$ 是 $0$。

\end{document}
%此后的内容会被忽略