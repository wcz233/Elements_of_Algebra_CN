\subsection{关于简单量的乘法}
\par
23. 当需要将两个或多个相等的数相加时, 可以简化其和的表达式。例如: 
\begin{gather*}
	a + a \ \text{等于}\  2 \times a, \\
	a + a + a \ \text{等于}\  3 \times a, \\
	a + a + a + a \ \text{等于}\  4 \times a, \text{依此类推}; 
\end{gather*}
其中 $\times$ 是乘法的符号。通过这种方式, 我们可以形成对乘法的概念, 并且需要注意: \par
\begin{gather*}
\begin{aligned}
&1 \times a \ \text{表示}\  2 \ \text{次}\  a, \ \text{或}\  a \ \text{的两倍}\ ;\\
&2 \times a \ \text{表示}\  3 \ \text{次}\  a, \ \text{或}\  a \ \text{的三倍}\ ;\\
&3 \times a \ \text{表示}\  4 \ \text{次}\  a, \ \text{或}\  a \ \text{的四倍}\ , \ \text{依此类推。}
\end{aligned}
\end{gather*}
\par
24. 因此, 如果用字母表示的一个数需要乘以另一个数, 只需将该数写在字母前。例如: \par
\begin{gather*}
\begin{aligned}
&a \ \text{乘以}\  20 \ \text{表示为}\  20a,\\
&b \ \text{乘以}\  30 \ \text{表示为}\  30b, \ \text{依此类推。}
\end{aligned}
\end{gather*}
显然, $1c$, 或 c 只取一次, 与 c 相同。
\par
25. 此外, 再将这些积乘以其他数也非常容易。例如: \par
\begin{gather*}
\begin{aligned}
&2 \ \text{次}\  3a, \ \text{或}\  3a \ \text{的两倍}\ , \ \text{是}\  6a;\\
&3 \ \text{次}\  4b, \ \text{或}\  4b \ \text{的三倍}\ , \ \text{是}\  12b;\\
&5 \ \text{次}\  7x \ \text{是}\  35x\ \text{。}
\end{aligned}
\end{gather*}
这些积还可以随意继续乘以其他数。
\par
26. 当要乘的数也用字母表示时, 将其直接写在另一个字母前。
例如, 将 $b$ 乘以 $a$, 积写作 $ab$; 
而 $pq$ 则是将 $q$ 乘以 $p$ 的积。
同样, 如果再将 $pq$ 乘以 $a$, 结果是 $apq$。
\par
27. 此处还可以进一步指出, 
字母相乘的顺序无关紧要。例如, $ab$ 与 $ba$ 相同, 
因为 $b$ 乘以 $a$ 与 $a$ 乘以 $b$ 是相同的。要理解这一点, 
只需将 $a$ 和 $b$ 替换为已知数字, 例如 $3$ 和 $4$; 
显而易见, $3$ 乘以 $4$ 与 $4$ 乘以 $3$ 是相同的。
\par
28. 不难看出, 当我们将数字代入这些连乘的字母表达式时, 
不能简单地将它们写在一起。例如, 如果将 $3$ 和 $4$ 相乘后写作 $34$, 
就得到了 $34$, 而不是 $12$。\par
因此, 当需要将普通数字相乘时, 
必须用乘号“ $\times$ ”或一个点“ $\cdot$ ”来分隔它们。\par
故,$3 \times 4,$ 或 $3 \cdot 4$ 意为 $3$ 乘 $4$, 也就是 $12$ ; 
同样, $1 \times 2$ 等于 $2$; $1 \times 2 \times 3$ 等于 $6$。\par
类似地 $1 \times 2 \times 3 \times 4 \times 56$ 等于 $1344$ , $1 \times 2 \times 3 \times 4 \times 5 \times 6 \times 7 \times 8 \times 9 \times 10$ 等于 $3628800$, 依此类推。
\par
29. 同样, 我们可以求出形如 $5 \cdot 7 \cdot 8 \cdot abcd$ 的表达式的值。
这表示 $5$ 必须乘 $7$, 然后将得到的积再乘 $8$; 
接下来, 将这三个数的积依次乘 $a$、$b$、$c$ 和 $d$。还可以注意到, 我们可以将 $5 \cdot 7 \cdot 8$ 替换为它们的积 $280$, 因为将 $5$ 乘 $7$ 的积 $35$ 再乘 $8$, 即得 $280$。
\par
30. 由两个或多个数的乘法所产生的结果称为\textbf{\textit{积(products)}}, 而这些数或单独的字母称为\textbf{\textit{因子(factors)}}。
\par
31. 迄今为止, 我们只考虑了正数的情况。毫无疑问, 这些正数所产生的积也是正的: $+b$ 乘以 $+a$ 必然得到 $+ab$。
但我们还需要分别研究 $+a$ 乘以 $-b$, 以及 $-a$ 乘以 $-b$ 时的结果。
\par
32. 让我们从 $-a$ 乘以 3 或 +3 开始。由于 $-a$ 可以看作是债务, 很明显, 如果将这笔债务算作三次, 
那么它将变为原来的三倍, 因此所得的积为 $-3a$。
同样, 如果 $-a$ 乘以 $+b$, 我们将得到 $-ba$, 或者换个形式, $-ab$。
因此, 我们可以得出结论: 如果一个正的数量与一个负的数量相乘, 所得的积是负的; 并且可以归纳出以下规则: \par
$+$ 乘以 $+$ 得 $+$ , \par
相反 $+$ 乘以 $-$ , \par
或 $\ \ \ -$ 乘以 $+$ 得 $-$。
\par
33. 接下来, 我们解决 $-$ 乘以 $-$ 的情况, 例如 $-a$ 乘以 $-b$。
首先从字母上看, 积显然是 $ab$, 但其前面的符号是 $+$ 还是 $-$ 尚不确定, 
我们只知道它必须是其中之一。现在, 我说它不可能是符号 $-$: 因为 $-a$ 乘以 $+b$ 得 $-ab$, 
而 $-a$ 乘以 $-b$ 不可能产生与 $-a$ 乘以 $+b$ 相同的结果, 而必须产生相反的结果, 即 $+ab$。
因此, 我们得出以下规则:\par
 $-$ 乘以 $-$ 得 $+$ , \par
 这与 $+$ 乘以 $+$ 的结果相同。
\par
34. 我们可以用以下简洁的方式表达这些规则: \par
\textbf{\textit{相同符号相乘, 得 $+$; 相异符号相乘, 得 $-$。}} \par
例如, 当需要将以下数相乘时: $+a$、$-b$、$-c$、$+d$, 我们首先计算 $+a$ 乘以 $-b$, 
得到 $-ab$; 接着, 用 $-c$ 乘以该积, 得到 $+abc$; 最后, 将 $+d$ 乘以该积, 得到 $+abcd$。
\par
35. 在解决符号问题后, 我们接下来要说明如何相乘本身是积的数。
例如, 如果将数 ab 乘以数 cd, 所得积为 abcd。这可以通过先将 ab 乘以 c, 然后再将所得结果乘以 d 来实现。
或者, 如果我们需要将 36 乘以 12, 由于 12 等于 3 乘以 4, 我们只需先将 36 乘以 3, 得 108, 
再将 108 乘以 4, 得出 36 乘以 12 的结果, 即 432。
\par
36. 如果我们想将 $5ab$ 乘以 $3cd$, 可以将其写作 $3cd \times 5ab$。
然而, 由于在此情况下乘数的顺序无关紧要, 按照惯例, 更好的做法是将普通数字写在字母前, 
并将积表示为 $5 \times 3abcd$, 即 $15abcd$, 因为 5 乘以 3 得 15。同样, 
如果我们需要将 $12pqr$ 乘以 $7xy$ , 我们会得到 $12 \times 7pqrxy$ , 即 $84pqrxy$ 。