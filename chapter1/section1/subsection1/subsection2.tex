\subsection{对加号“ $+$ ”和减号“ $-$ ”的解释}


8. 当我们需要将一个给定的数字加到另一个数字上时, 
	用符号“ $+$ ”表示, 这个符号放在第二个数字前面, 读作“加”。
	例如, $5 + 3$ 表示要将 $3$ 加到 $5$ 上, 在这种情况下, 众所周知结果是 $8$。
	同样, $12 + 7$ 等于 $19$; $25 + 16$ 等于 $41$; $25 + 41$ 的和是 $66$, 等等。
\par
9. 我们也使用相同的加号“$+$”来连接几个数字。
	例如, $7 + 5 + 9 + 9$ 表示将 $5$ 和 $9$ 分别加到 $7$ 上, 
	结果是 $21$。因此, 读者应该能够理解以下表达式的含义: 
\[
		8 + 5 + 13 + 11 + 1 + 3 + 10
\]\par
	即这些数字的总和是 $51$。
\par
10. 这一切都显而易见; 我们只需提到, 在代数学中, 为了对数字进行泛化, 	
	我们用字母来表示它们, 如 $a$、$b$、$c$、$d$ 等。
	因此, 表达式 $a + b$ 表示由 a 和 b 代表的两个数字的总和, 
	而这些数字可以非常大, 也可以非常小。同样, 
$f + m + b+ x$
表示由这些四个字母代表的数字的总和。
\par
	因此, 如果我们知道这些字母所代表的数字, 就可以随时通过算术来找到这些表达式的总和或值。
\par
11. 相反, 当需要从一个给定数字中减去另一个数字时, 这个操作用符号“$-$”表示, 读作“减”,
 并放在要被减去的数字前。例如, $8 - 5$ 表示从 $8$ 中减去 $5$, 结果是 $3$。同样, $12 - 7$ 等于 $5$; $20 - 14$ 等于 $6$, 等等。
\par
12. 有时, 我们可能需要从一个数字中减去多个数字。例如: 
\[
	50 - 1 - 3 - 5 - 7 - 9
\]
这表示首先从 $50$ 中减去 $1$, 剩下 $49$; 然后从这个余数中减去 $3$, 剩下 $46$; 
再减去 $5$, 剩下 $41$; 减去 $7$, 剩下 $34$; 最后, 从中减去 $9$, 剩下 $25$: 这个最终的余数就是该表达式的值。
但由于 $1$、$3$、$5$、$7$、$9$ 这几个数字都要被减去, 因此可以直接将它们的总和 $25$ 一次性从 $50$ 中减去, 结果仍然是 $25$。
\par
13. 对于包含加号“$+$”和减号“$-$”的类似表达式, 其值也容易确定。例如: 
\[
	12 - 3 - 5 + 2 - 1 = 5
\]
	我们只需分别计算带有加号“$+$”的数字的总和, 
	并从中减去带有减号“$-$”的数字的总和。
	因此, $12$ 和 $2$ 的总和是 $14$; 
	而 $3$、$5$ 和 $1$ 的总和是 $9$; 
	于是, 从 $14$ 中减去 $9$, 剩下 $5$。
\par
14. 从这些例子可以看出, 
	只要保留每个数字前面的正确符号, 
	数字的书写顺序完全无关紧要且是任意的。
	我们可以同样正确地将前一条中的表达式排列为: \par
	   $12 + 2 - 5 - 3 - 1$, \par
	或 $2 - 1 - 3 - 5 + 12$, \par
	或 $2 + 12 - 3 - 1 - 5$, \par
	或者其他不同的顺序; 在最初的排列中, 需要注意, 假定数字 $12$ 前面有一个加号“$+$”。
\par
15. 如果为了将这些运算一般化, 我们使用字母代替实际数字, 这并不会带来更多困难。例如, 很明显, 表达式
\[
	a - b - c + d - e
\]
	表示我们有由 $a$ 和 $d$ 表示的数字, 
	而需要从这些数字或它们的和中减去由字母 $b$、$c$ 和 $e$ 表示的数字, 
	它们前面带有符号“$-$”。
\par
16. 因此, 必须严格注意每个数字前所带的符号。
	在代数学中, 简单数量是考虑其前面所带符号的数字。
	进一步说, 我们称那些前面带有符号“$+$”的量为正量, 
	而那些受符号“$-$”影响的量为负量。
\par
17. 我们通常计算一个人财产的方法恰好说明了上述情况。
	我们用正数来表示一个人实际拥有的财产, 
	并使用或默认加号“$+$”; 
	而他的债务则用负数表示, 
	或者使用符号“$-$”。
	因此, 当说某人拥有 $100$ 克朗, 
	但欠债 $50$ 克朗时, 
	这意味着他的实际财产是 $100 - 50$; 或者换个形式, $+100 - 50$, 即 $50$。
\par
18. 由于负数可以被视为债务, 而正数代表实际拥有的财产, 我们可以说负数小于零。
因此, 当一个人一无所有却欠 $50$ 克朗时, 显然他比一无所有还要少 $50$ 克朗。
因为即使有人赠送他 $50$ 克朗来偿还债务, 他也只是达到零的水平, 尽管实际上比以前富有。
\par
19. 因此, 正数无可争议地大于零, 而负数则小于零。我们通过在 $0$ 上加 $1$ 来得到正数, 即在零的基础上加 $1$; 
并不断从单位数开始增加。这就是所谓自然数序列的起源, 其主要项如下: 
\[
0, +1, +2, +3, +4, +5, +6, +7, +8, +9, +10, \ \text{依此类推至无穷。}
\] 
但如果我们不是通过连续加 $1$ 来延续这一序列, 而是反方向进行, 即不断减去 $1$, 就会得到以下负数序列: 
\[
0, -1, -2, -3, -4, -5, -6, -7, -8, -9, -10, \ \text{依此类推至无穷。}
\] 
20. 这些数字, 无论是正数还是负数, 都被统称为整数, 它们因此要么大于零, 要么小于零。
我们称它们为整数, 是为了将其与分数以及其他一些种类的数字区分开来, 这些将在后面讨论。
例如, $50$ 比 $49$ 多 $1$ 个完整的单位, 很容易理解, 在 $49$ 和 $50$ 之间可能有无数个中间数, 
这些数都大于 $49$, 但又都小于 $50$。我们只需想象两条线, 
一条长 $50$ 英尺, 另一条长 $49$ 英尺, 很明显, 可以画出无数条线, 它们都比 $49$ 英尺长, 但又比 $50$ 英尺短。
\par
21. 在整个代数学中, 对负量的准确理解极为重要。然而, 我在这里仅指出, 所有如下形式的表达式: \par
	$+1 - 1, +2 - 2, +3 - 3, +4 - 4$, 等等\par
	都等于 $0$, 或零。而表达式
	$+2 - 5$ 等于 -3, 
	因为如果一个人有 $2$ 克朗, 
	但欠 $5$ 克朗, 他不仅一无所有, 还欠 $3$ 克朗。\par
	同样, $7 - 12$ 等于 $-5$, 
	而 $25 - 40$ 等于 $-15$。  
\par
22. 同样的观察结论也适用于使用字母代替数字以使表达式更加一般化的情况。
	例如, $+a - a$ 的值始终为 $0$ 或零。但如果我们想知道 $+a - b$ 的值, 需要考虑两种情况。\par
	第一种情况是 $a$ 大于 $b$, 此时应从 $a$ 中减去 $b$, 
	其余数(其前面带有符号“$+$”或默认带有符号“$+$”)即为所求的值。\par
	第二种情况是 $a$ 小于 $b$, 此时应从 $b$ 中减去 $a$, 
	将余数变为负数, 在其前面加上符号“$-$”, 即为所求的值。  
\par
