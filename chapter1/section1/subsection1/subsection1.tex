\subsection{关于数学的一般概述}


1. 凡是能够增加或减少的东西, 称为量或数量。
因此, 金钱是一种数量, 因为我们可以增加或减少它。
重量以及其他类似性质的事物也是如此。


2. 根据这一定义, 很明显, 不同种类的量必然非常多, 
以至于难以一一列举。这就是数学各个分支的起源, 每个分支都专注于某种特定类型的量。
数学总体上是研究数量的科学, 或者说, 是研究如何测量数量的科学。


3. 现在, 我们无法测量或确定任何数量, 除非先将同类中的某个已知数量作为参照, 并指出它们之间的相互关系。
例如, 如果要确定一笔钱的数量, 我们需要选择某种已知的货币单位, 
如路易金币、克朗、杜卡特或其他硬币, 然后计算这笔钱中包含多少个这样的单位。
同样, 如果要确定重量的数量, 我们需要选择某个已知的重量单位, 
如磅或盎司, 然后计算这种单位在我们试图确定的重量中包含多少次。
如果我们希望测量某种长度或距离, 则可以使用已知的长度单位, 比如一英尺。


4. 因此, 各种类型的量的确定或测量可以归结为以下方法:任意选择与待测量同类的一已知量, 
将其作为标准或单位;然后, 确定待测量与该已知标准之间的比例。
这个比例总是用数字表示, 因此, 数字只是一个量与另一个任意设定为单位的量之间的比例。


5. 由此可见, 所有的量都可以用数字表示, 
并且所有数学科学的基础必须建立在对数字科学的完整研究和对各种可能计算方法的精确考察之上。
这一数学的基础部分被称为 \textbf{\textit{分析学(Analysis)}} 或 \textbf{\textit{代数学(Algebra)}} 。

