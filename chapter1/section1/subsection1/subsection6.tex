\subsection{整数在其因数方面的性质}
\par
58. 如前所述, 有些数可以被某些因数整除, 而另一些则不能。
为了更深入地了解整数之间的区别, 我们需要仔细观察这种差异。
这不仅包括区分能被因数整除的数与不能被整除的数, 还包括研究后者在除法中所余的余数。为此, 我们将依次考察以下因数: 
\[
	2, 3, 4, 5, 6, 7, 8, 9, 10, \dots
\]
59. 首先, 以$2$为因数。能被$2$整除的数有: 
\[
	2, 4, 6, 8, 10, 12, 14, 16, 18, 20, \ \text{等等。}\  
\]
这些数显然总是以$2$为增量递增。这些数被称为\textbf{\textit{偶数(even numbers)}}。  
然而, 还有一些数, 如: 
\[
1, 3, 5, 7, 9, 11, 13, 15, 17, 19, \ \text{等等}\ , 
\]
它们与上述偶数始终相差$1$, 且无法被$2$整除, 而在除以$2$时总余$1$, 这些数被称为\textbf{\textit{奇数(odd numbers)}}。
\par
偶数可以用通用表达式 $2a$ 表示, 因为将整数 $1, 2, 3, 4, 5, 6, 7, $等依次代入 $a$, 即可生成所有偶数。
因此, 奇数可以用通式 $2a + 1$ 表示, 因为 $2a + 1$ 比偶数 $2a$ 大1。
\par
60. 其次, 设因数为3。能被3整除的数有: 
\[
	3, 6, 9, 12, 15, 18, 21, 24, 27, 30, \ \text{等等。}\ 
\]
这些数可以用表达式 $3a$ 表示, 因为 $3a$ 除以$3$时, 商为 $a$, 且没有余数。  
其他不能被$3$整除的数在除以$3$时会余$1$或$2$, 因此分为两类:   
第一类, 除以$3$余$1$的数有: 
\[
	1, 4, 7, 10, 13, 16, 19, \ \text{等等}\ , 
\]
可以用 $3a + 1$ 表示; 
第二类, 除以$3$余$2$的数有: 
\[
	2, 5, 8, 11, 14, 17, 20, \ \text{等等}\ , 
\]
可以用 $3a + 2$ 表示。  
因此, 所有整数都可以用以下三种形式之一表示: 
$3a$、$3a + 1$或$3a + 2$。

61. 再次, 设因数为$4$。能被$4$整除的数有: 
\[
	4, 8, 12, 16, 20, 24, \ \text{等等}\ , 
\]
这些数以$4$为增量递增, 可以用表达式 $4a$ 表示。  
而其他不能被$4$整除的数在除以$4$时, 可以有以下三种余数:   
第一类, 余数为$1$的数有: 
\[
	1, 5, 9, 13, 17, 21, 25, \ \text{等等}\ , 
\]
可以用 $4a + 1$ 表示; 
第二类, 余数为$2$的数有: 
\[
	2, 6, 10, 14, 18, 22, 26, \ \text{等等}\ , 
\]
可以用 $4a + 2$ 表示;   
第三类, 余数为$3$的数有: 
\[
	3, 7, 11, 15, 19, 23, 27, \ \text{等等}\ , 
\]
可以用 $4a + 3$ 表示。  

因此, 所有整数都可以用以下四种形式之一表示: 
\[
	4a, 4a + 1, 4a + 2, 4a + 3.
\]
62. 当以$5$为因数时, 情况与之前的讨论大致相同。所有能被$5$整除的数都可以用表达式 $5a$ 表示, 而不能被$5$整除的数则可以归为以下四种形式之一:   
\[
	5a + 1, 5a + 2, 5a + 3, 5a + 4.   
\]
同样的方法可以应用于更大的因数。

63. 这里有必要回顾一下关于将数分解为其简单因数的内容。  
任何包含因数$2$、$3$、$4$、$5$、$7$或其他数的整数, 必然能被这些数整除。例如, $60$可以表示为 $2 \times 2 \times 3 \times 5$, 因此显然$60$可以被$2$、$3$和$5$整除。

64. 此外, 通式 $abcd$ 不仅可以被 $a$、$b$、$c$ 和 $d$ 整除, 还可以被以下任意因子组合整除: \par
两个因子的组合: $ab$、$ac$、$ad$、$bc$、$bd$、$cd$; \par
三个因子的组合: $abc$、$abd$、$acd$、$bcd$; \par
以及所有因子的组合 $abcd$, 也就是其自身。\par
由此可见, $60$(即 $2 \times 2 \times 3 \times 5$)不仅可以被这些简单因数整除, 
还可以被任意两个简单因数的乘积整除, 如$4$、$6$、$10$、$15$; 也可以被任意三个因数的乘积整除, 如$12$、$20$、$30$; 最后, 还可以被$60$自身整除。

65. 因此, 当我们将一个任意选定的数分解为其简单因数后, 便可以轻松列出所有能整除它的数。  
具体方法是: 
首先将这些简单因数逐一列出; 然后将它们两两相乘, 接着三三相乘, 四四相乘, 依此类推, 直到得到该数本身。

66. 此处需要特别注意以下两点:
\begin{gather*}
\begin{aligned}
	&\ \text{每个数都可以被}\ 1\ \text{整除}\ ; \\
	&\ \text{每个数都可以被它本身整除。}\ 
\end{aligned}
\end{gather*}
因此, 每个数至少有两个因数(factors)或约数(divisors), 即该数本身和\textbf{\textit{1(unity)}}。
然而, 对于那些除了$1$和它本身之外没有其他因数的数, \par
我们将其归为\textbf{\textit{简单数(simple)}}或\textbf{\textit{素数(prime numbers)}}。  

除这些素数外, 其他所有数除了$1$和自身之外, 还具有其他因数。这可以从以下表格中看出, 在每个数下面列出了它的所有因数: 
\begin{table}[H]
	\centering
	\begin{tblr}{
	  row{odd} = {c},
	  row{2} = {c,t},
    vlines = {},
	  hlines = {}
	}
	1  & 2      & 3      & 4         & 5      & 6            & 7      & 8            & 9         & 10            & 11      & 12                  & 13      & 14            & 15            & 16               & 17      & 18                  & 19      & 20                   \\
	1  & {1\\2} & {1\\3} & {1\\2\\4} & {1\\5} & {1\\2\\3\\6} & {1\\7} & {1\\2\\4\\8} & {1\\3\\9} & {1\\2\\5\\10} & {1\\11} & {1\\2\\3\\4\\6\\12} & {1\\13} & {1\\2\\7\\14} & {1\\3\\5\\15} & {1\\2\\4\\8\\16} & {1\\17} & {1\\2\\3\\6\\9\\18} & {1\\19} & {1\\2\\4\\5\\10\\20} \\
	1  & 2      & 2      & 3         & 2      & 4            & 2      & 4            & 3         & 4             & 2       & 6                   & 2       & 4             & 4             & 5                & 2       & 6                   & 2       & 6                    \\
	P. & P.     & P.     &           & P.     &              & P.     &              &           &               & P.      &                     & P.      &               &               &                  & P.      &                     & P.      &                      
	\end{tblr}
\end{table}
表中以 “P.” 标注的为素数。

67. 最后, 需要注意的是, $0$, 或\textbf{\textit{零(nothing)}}, 可以被视为一个具有特殊性质的数: 它能被所有可能的数整除。  
无论用何数 $a$ 去除$0$, 其商始终为$0$。因为任何数乘以\textbf{\textit{零(nothing)}}都会得到零, 所以 $0 \times a$ 或 $0a$ 是 $0$。