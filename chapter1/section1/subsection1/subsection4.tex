\subsection{关于整数的性质及其因子}
\par
37. 我们已经指出, 积是由两个或多个乘数相乘得到的, 
这些乘数被称为因子。因此, 数字 $a$、$b$、$c$、$d$ 是积 $a$bc$d$ 的因子。
\par
38. 如果我们将所有整数视为由两个或多个数相乘而成的积, 那么很快就会发现, 
有些数不能由这样的乘法得出, 因此没有任何因子; 而另一些数可以由两个或多个数相乘得出, 
因此有两个或多个因子。例如, $4$ 是由 $2 \times 2$ 生成的; $6$ 是由 $2 \times 3$ 生成的; 
$8$ 是由 $2 \times 2 \times 2$ 生成的; $27$ 是由 $3 \times 3 \times 3$ 生成的; 
$10$ 是由 $2 \times 5$ 生成的, 等等。
\par
39. 另一方面, 数字 $2$、$3$、$5$、$7$、$11$、$13$、$17$ 等不能以相同的方式用因子表示, 
除非我们使用 $1$, 例如将 $2$ 表示为 $1 \times 2$。但由于任何数与$1$相乘仍为其自身, 因此不应将$1$视为因子。  
因此, 所有不能用因子表示的数字, 
例如 $2$、$3$、$5$、$7$、$11$、$13$、$17$ 等, 
被称为\textbf{\textit{素数(simple)}} 或 \textbf{\textit{质数(prime numbers)}}; 
而其他可以用因子表示的数字, 例如 $4$、$6$、$8$、$9$、$10$、$12$、$14$、$15$、$16$、$18$ 等, 
被称为\textbf{\textit{合数(composite)}}。
\par
40. 质数或素数由于不是由两个或多个数相乘得到的, 因此值得特别注意。
还有一点特别值得关注的是, 如果按顺序写出这些质数, 如下: 
\[
2, 3, 5, 7, 11, 13, 17, 19, 23, 29, 31, 37, 41, 43, 47, \ \text{等等, }
\]
我们无法发现任何规律; 它们之间的差距有时较大, 有时较小。
至今尚无人能够确定它们是否遵循某种特定的规律。
\par
41. 所有可以用因子表示的合数, 都是由上述质数生成的。也就是说, 
它们的所有因子都是质数。例如, 如果我们找到一个不是质数的因子, 
它总是可以被分解为两个或多个质数。例如, 将数字 $30$ 表示为 $5 \times 6$, 
但由于 $6$ 不是质数, 而是由 $2 \times 3$ 生成的, 我们可以将 $30$ 表示为 $5 \times 2 \times 3$, 
或 $2 \times 3 \times 5$, 即完全由质数组成的因子。
\par
42. 如果我们现在考虑那些可以分解为质因子的合数, 会发现它们之间存在很大的差异。
例如, 我们会发现有些数只有两个因子, 有些数有三个因子, 还有更多的有更多因子。例如:   
\begin{gather*}
	\begin{aligned}
		4 &= 2 \times 2, \\
		6 &= 2 \times 3, \\
		8 &= 2 \times 2 \times 2, \\
		9 &= 3 \times 3, \\
		10 &= 2 \times 5, \\
		12 &= 2 \times 3 \times 2, \\
		14 &= 2 \times 7, \\
		15 &= 3 \times 5, \\
		16 &= 2 \times 2 \times 2 \times 2, \\
	\end{aligned}
\end{gather*}
等等。
\par
43. 因此, 可以很容易地找到一种方法来分析任意数字或将其分解为其质因子。
例如, 给定数字 $360$, 我们首先将其表示为 $2 \times 180$。\par
现在, $180$ 等于 $2 \times 90$, 而  
\begin{gather*}
	\begin{aligned}
		90 &= 2 \times 45 \\
		45 &= 3 \times 15 \\
		15 &= 3 \times 5 \\
	\end{aligned}
\end{gather*}
所以, 数字 $360$ 可以表示为这些质因子的乘积: $2 \times 2 \times 2 \times 3 \times 3 \times 5$, 
因为这些数相乘的结果是 $360$。

44. 这表明质数不能被其他数字整除; 
另一方面, 复合数的质因子可以通过寻找能将这些复合数整除的质数来最方便、最可靠地找到。
而要做到这一点, 就需要用到\textbf{\textit{除法(Division)}}。因此, 我们将在下一章中解释这一操作的规则。

