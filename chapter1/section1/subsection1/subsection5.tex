\subsection{关于简单量的除法}
\par
45. 当将一个数分成两份、三份或更多相等部分时, 
可以通过\textbf{\textit{除法(div-~ision)}}来完成, 
这使我们能够确定其中一个部分的大小。
例如, 当我们希望将数字 $12$ 分成三份相等的部分时, 通过除法我们得知每部分的大小为 $4$。
在这种运算中使用以下术语: 要被分解或除去的数称为\textbf{\textit{被除数(dividend)}}; 
要找的那个大小的数量称为\textbf{\textit{除数(divisor)}}; 
通过除法确定的其中一份的大小称为\textbf{\textit{商(quotient)}}。因此, 在上述例子中: 
\begin{gather*}
	\begin{aligned}
12 \ \text{是被除数, }  
3 \ \text{是除数, } 
4 \ \text{是商。}
	\end{aligned}
\end{gather*}

46. 由此可见, 如果将一个数除以 $2$, 
或者将其分成两等份, 则其中一份(即商)取两次就正好等于原来的数。
同样地, 如果将一个数除以 $3$, 那么商取三次也会得到原来的数。
一般来说, 将商乘以除数, 必然能够重新得到被除数。

47. 正因如此, \textbf{\textit{除法(division)}}被称为一种能够帮助我们找到一个数量(即商)的运算规则, 
这个数乘以除数可以准确地得到被除数。
例如, 如果将 $35$ 除以 $5$, 就是说我们需要找到一个数, 这个数乘以 $5$ 会得出 $35$。
这个数是 $7$, 因为 $5$ 乘 $7$ 等于 $35$。这种推理的表达方式是: “$7$ 在 $35$ 中能包含 $5$ 次”, 并且“$5$ 乘 $7$ 等于 $35$”。

48. 因此, 被除数可以看作是一个乘积, 其中一个因子是除数, 
另一个因子是商。例如, 如果我们要将 $63$ 除以 $7$, 我们试图找到一个乘积, 
使得 $7$ 作为其中一个因子, 且另一个因子乘以 $7$ 时恰好等于 $63$ 。此时 $7\times9$ 恰好是这样的乘积, 
而 $9$ 也正是 $63$ 除以 $7$ 时得到的商。

49. 一般而言, 如果我们将数字 $ab$ 除以 $a$, 很明显商是 $b$, 因为 $a$ 乘 $b$ 得到被除数 $ab$。
同样地, 如果我们将 $ab$ 除以 $b$, 商就是 $a$。在所有可以提出的除法例子中, 如果用被除数除以商, 
我们将再次得到除数。例如, $24$ 除以 $4$ 得 $6$, 同样 $24$ 除以 $6$ 又得 $4$。

50. 由于整个运算过程实质上是将被除数表示为两个因子的乘积, 其中一个因子等于除数, 
另一个因子等于商, 以下例子将更容易理解。我说, 首先, 
被除数 $abc$ 除以 $a$, 得到的商是 $bc$, 因为 $a$ 乘 $bc$ 得到 $abc$。同样地, $abc$ 除以 $b$, 
商是 $ac$, 并且 $abc$ 除以 $ac$ 将得到 $b$。
很明显, $12mn$ 除以 $3m$ 得到 $4n$; $3m$ 乘以 $4n$ 得到 $12mn$。
而若同样这个数 $12mn$ 除以 $12$ , 我们也将得到商 $mn$。

51. 由于每个数 $a$ 都可以表示为 $1a$ 或 $a$, 很明显, 如果我们将 $a$ 或 $1a$ 除以 $1$, 
商将是相同的数 $a$。相反地, 如果将相同的数 $a$ 或 $1a$ 除以 $a$, 商将是 $1$。

52. 然而, 有时我们无法将被除数表示为两个因子的乘积, 其中一个因子等于除数。在这种情况下, 无法按照我们描述的方法完成除法运算。
例如, 将 $24$ 除以 $7$, 很明显 $7$ 不是 $24$ 的因子, 因为 $7\times3$ 仅等于 $21$, 
小于 $24$, 而 $7\times4$ 等于 $28$, 大于 $24$。
因此, 可以得出结论, 那个商必须大于 $3$ 且小于 $4$。为了更准确地确定商, 
我们需要使用另一种数——\textbf{\textit{分数(fraction)}}, 我们将在后续章节中讨论。

53. 在使用分数之前, 通常情况下, 人们会取最接近真实商的整数, 同时也关注留下的余差部分。
因此, 我们说, $7$ 在 $24$ 中能包含 $3$ 次, 残留部分是 $3$, 因为 $7$ 乘以 $3$ 仅得 $21$, 比 $24$ 小 $3$。  
类似地, 可以用同样的方式考虑以下例子:   
\[
\longdiv{34}{6}
\]
也就是说, 除数是$6$, 被除数是$34$, 被除数是$5$, 余数是$4$。
\[
\longdiv{41}{9}
\]
这里除数是$9$, 被除数是$41$, 商是$4$, 余数是$5$。
\par
在有余数的例子中,应遵守以下规则。

54. 用商乘以除数, 然后将余数加到所得的积上, 结果应等于被除数。
这种方法可以用来验证除法计算是否正确。例如, 在上述第一个例子中, 
将 $6$ 乘以 $5$ 得 $30$, 再加上余数 $4$, 结果是 $34$, 即被除数。在第二个例子中, 
将除数 $9$ 乘以商 $4$ 得 $36$, 再加上余数 $5$, 结果是 $41$, 即被除数。

55. 最后, 需要注意关于符号 \textbf{\textit{“$+$”正号(plus)}} 和 \textbf{\textit{“$-$”负号(minus)}}的问题。
如果将 $+ab$ 除以 $+a$, 商是 $+b$, 这是显然的。如果将 $+ab$ 除以 $-a$, 商是 $-b$, 因为 $-a$ 乘以 $-b$ 得到 $+ab$。同样地, 
如果将 $-ab$ 除以 $+a$, 商是 $-b$, 因为 $+a$ 乘以 $-b$ 得到 $-ab$。最后, 如果将 $-ab$ 除以 $-a$, 商是 $+b$, 因为 $-a$ 乘以 $+b$ 得到 $-ab$。

56. 因此, 对于符号 “$+$” 和 “$-$”, 除法遵循与乘法相同的规则, 即: 
\begin{gather*}
\begin{aligned}
&+ \ \text{除以}\  + \ \text{得}\  +; \\  
&+ \ \text{除以}\  - \ \text{得}\  -; \\
&- \ \text{除以}\  + \ \text{得}\  -; \\
&- \ \text{除以}\  - \ \text{得}\  +.
\end{aligned}
\end{gather*}
简而言之, 相同符号相除得正, 不同符号相除得负。

57. 例如, 将 $18pq$ 除以 $-3p$, 商是 $-6q$。此外,   
\begin{gather*}
\begin{aligned}
 -30xy \ \text{除以} +6y, \ \text{得} -5x;   
 -54abc \ \text{除以} -9b, \ \text{得} +6ac;   
\end{aligned}
\end{gather*}
在最后一个例子中, $-9b$ 乘以 $+6ac$ 等于 $-54abc$。  
关于简单量的除法已讨论足够多, 接下来我们将在补充一些关于数的性质及其因子的说明后, 转而解释\textbf{\textit{分数(fractions)}}。
